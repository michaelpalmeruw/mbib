% LaTeX documentation for mbib. It loads some of my personal packages;
% therefore, this documentation won't compile out of the box after
% cloning the repository. 
%
% The screenshot console windows were, for the most part, 80x25 characters 
% big. console window font size was 16 (Consolas)

\documentclass[10pt]{article}

\usepackage[margin={0.5in,0.75in},papersize={7in,10in}]{geometry}
\usepackage[scale=0.92]{mybookfonts}
\usepackage{xspace,verbatim,graphicx,shortvrb}

\usepackage[small]{titlesec}
\setlength{\parindent}{2em}
\usepackage{simplehref, vf-plaindoc, setspace}
\setstretch{1.2}
%\onehalfspacing

\newcommand*{\mbib}{\texttt{mbib}\xspace}
\newcommand*{\jabref}{\texttt{JabRef}\xspace}
\newcommand*{\loo}{\texttt{Libre\-/Open\-office}\xspace}
\newcommand*{\bibtex}{\texttt{BibTex}\xspace}
\newcommand*{\sqlite}{\texttt{SQLite}\xspace}
\newcommand*{\ini}{\texttt{.mbib.ini}\xspace}

\newcommand*{\key}[1]{\texttt{#1}\xspace}
\newcommand*{\arrowdown}{\key{$\downarrow$}}
\newcommand*{\arrowup}{\key{$\uparrow$}}
\newcommand*{\arrowright}{\key{$\rightarrow$}}

\newcommand{\screenshot}[2][]{%
\medskip\par
\begin{center}
%\includegraphics[width=1300px,#1]{/data/code/mbib/docs/screenshots/#2}
\includegraphics[#1]{/data/code/mbib/docs/screenshots/#2}
\end{center}}

\MakeShortVerb{\~}

\begin{document}

\section*{\mbib Documentation}

\emph{As of \today, the tutorial section is nearing completion, but the reference section is still missing.}

\bigskip

\noindent \mbib is a literature manager, with capabilities similar to \href{http://www.jabref.org}{\jabref}, but intended to better cope with large databases containing thousands of references.  Key features and limitations are:

\begin{itemize}
\item Import from and export to \bibtex  

\item Import records from DOI and PubMed identifiers

\item Push citations to \texttt{Texmaker}/\texttt{TexStudio} and \loo 

\item Written in Python3

\item Console-based GUI, based on the \texttt{urwid} and
\texttt{urwidtrees} Python libraries. 

\item Data are stored in a \sqlite database

\item Developed and tested only on Linux; will most likely not work out of the box on other platforms (volunteers for porting it to other platforms are welcome)

\end{itemize}

\section{Preliminary notes}

\subsection{Motivation}

I started writing \mbib after my previous literature manager (\texttt{bibus}) became unusable because of growing incompatibilities with LibreOffice, wxPython etc. I tried using \jabref for a while, and while it's really pretty good in many ways, it bogs down once you have several thousand references in your database. (The same goes for Zotero and Mendeley.)

\medskip

\noindent My main objectives for \mbib were 

\begin{itemize}
\item Keep all references in a single database
\item Use a proper database engine for flexibility and performance
\item Keep the database structure simple to allow direct querying and manipulation with SQL
\item Work seamlessly with both \LaTeX\ (my tool of choice) and 
\item \loo (which I use when I have to) 
\end{itemize}

While I would like the interaction with \loo to be a little smoother, making it so would take substantial effort; otherwise, these objectives have been met.

%The database has a fairy simple structure that lends itself to direct querying and manipulation with SQL
%
%Therefore, starting the program does \emph{not} load the entire database into memory. 

\subsection{Development status}

At present, \mbib is alpha software, yet production-ready ;-) -- it has almost all the features I need, but incompatible changes may still happen to the code, the configuration, and even the database structure. However, in the latter case, I will provide a script for migrating the database to the new format (which I will have to do with my own database anyway). 

The program was written primarily with my own needs in mind. I work in biochemistry and use PubMed as my main online literature source, and I have therefore given interaction with that database priority. People in other fields might miss tighter integration with other databases. I'm open to adding support for those, but likely won't do so unless prodded. Similarly, some other bits of functionality are tailored to my own personal preferences and may seem a little narrow or idiosyncratic to other users. Let me know if you need or want to contribute enhancements. 

The code violates all manner of software engineering gospel. There are no unit tests, and I don't plan to add them; the doc strings are a bit spotty and not formatted for automatic conversion into API docs. That said, I will give an overview of the program structure below, which hopefully will help you find your way through the code. 

\section{Installation}

In the following, I am going to describe how things work on \emph{my} system. I am running Debian with a KDE desktop. I don't suppose there will be any major differences with other Linux distros or window managers, but I am not going to verify this by experiment. If you manage to get it to work on other systems and have some specific tricks to share, please let me know,%
%
\footnote{mpalmeruw@gmail.com}
%
and I will include them here.

\subsection{Prerequisites}

In order to run \mbib, you first need to install these programs and libraries:

\begin{itemize}
\item Bash

\item Python3 

\item \sqlite

\item The \texttt{urwid}, \texttt{urwidtrees}, \texttt{lxml}, and \texttt{requests} libraries for Python3

\item To use \mbib with \loo, you also need the \texttt{PyUNO} bridge for Python3. For advanced bibliography formatting in \loo, you will also need \jabref

\item If you want to copy items to the X clipboard, you will need \texttt{xsel}

\item For viewing or emailing PDF files, you will need \texttt{xdg-open} and \texttt{xdg-email}
% anything else?
\end{itemize}

\noindent Bash is probably present on any Linux system by default. On Debian, all other prerequisites can be installed through the system's package manager. A copy of \sqlite already comes as part of the standard library when you install Python3, but you may also want to install the \texttt{sqlite3} package, which provides the command line client that lets you run SQL statements directly on your database. 

The \texttt{xdg-open} and \texttt{xdg-email} utilities may already be installed by default on your graphical Linux desktop; in Debian, they reside in the \texttt{xdg-utils} package.

\subsection{Installing \mbib}

Clone (or download and unzip) the \texttt{git} repository into a convenient location. Add the \mbib installation directory (\texttt{mbib-master}) to your bash \texttt{\$PATH}.

\subsection{Configuration}

The first program start will generate a configuration \ini file in a default location. The available settings are explained in comments inside the file itself.%
%
\footnote{You can have multiple configuration files and select one on the command line. This is explained in section \dots}

\section{Tutorial}
\label{sec:tutorial}

Here we give an overview of the general work flow. We assume that the default configuration settings are in effect. A reference section describing all program features and configuration options will follow later. 

\subsection{Starting the program}

Assuming you have installed all prerequisites and added the \mbib installation directory to your shell's \texttt{\$PATH}, you should now be able to open a console window and run 

\begin{verbatim}
mpalmer@rehakles:~$ mbib.sh
\end{verbatim}

\noindent After the first start, the expected output should be 

\begin{verbatim}
Config file (...)/mbib/data/mbib.ini not found!
Create default configuration file and proceed (1) or exit (2)?
1) proceed
2) exit
#? 
\end{verbatim}

\noindent Press \key{1} and \key{Enter}, and you should see

\begin{verbatim}
#? 1
OK
Database file (...)/mbib/data/mbib.sqlite not found. Create? (y/n) 
\end{verbatim}

\noindent Press \key{y} and the program should start. The interface should look like this:

\screenshot{initial-screen}

\noindent \mbib displays all references in a tree structure. The main \texttt{References} folder at this time has two sub-folders. Move between the displayed items using the arrow keys, the mouse wheel, or click directly on the desired folder. Go to the ``general interest'' folder and press \key{F2}. You should now see the references contained in this folder:

\screenshot{f2-pressed}

\noindent Pressing \key{F2} once more will close the folder again. However, for now leave it open, and use the arrow keys or the mouse again to highlight the record with the label  \texttt{Harrit2009} (which by the way also functions as its \bibtex key). 

\subsection{Working with references}

When \texttt{Harrit2009} is highlighted, press \key{Enter} (or use a double mouse click), and you should see this context menu that presents all operations available for a single reference:

\screenshot{reference-menu}

\noindent Navigate the menu using \arrowup, \arrowdown and select an item with \key{Enter}, or alternatively use the indicated shortcut keys or a single mouse click. If we select \texttt{Show details}, the \texttt{View reference} dialog comes up:

\screenshot{refview}

\noindent Use the \arrowup/\arrowdown keys again to scroll through this dialog.  The abstract is at the bottom. Above it, there are three fields that are hyperlinks. The one currently in focus is highlighted in blue; you can activate it by pressing \key{Enter} or again using the mouse, which will open a browser window and take you to the corresponding URL. 

Press \texttt{Esc} to close the dialog, and then \key{Enter} to reopen the menu. This time, choose \texttt{Edit details}. The dialog will look similar, but now the content of all fields is editable, and fields that are currently empty are displayed as well. Make some changes---for example, add a comment: ``That stuff they found in the dust must of been paint peeled off from them box cutters.'' Press \texttt{Tab} to switch to the buttons at the bottom; use the left and right arrow keys to select ``Cancel'' or ``OK'', and \key{Enter} to confirm or abort the edit.  

\subsection{Moving references around}
\label{sec:moving}

The paper we are looking at presents clear scientific evidence demonstrating that the WTC towers were rigged with explosives. This notion might disturb you, and you might therefore want to delete this reference. To do so, bring up the menu again and press \texttt{d} to delete the reference from this one folder, or \texttt{r} to also delete any copies residing in other folders (which in this case don't exist). After confirmation, the reference now shows up in the Trash folder:

\screenshot{harrit-trash}

\noindent Navigate down to the Trash folder and press \key{Enter} or double-click on it to bring up this folder's context menu:

\screenshot{trash-menu}

\noindent Selecting the ``Empty Trash'' option will get rid of the deleted reference entirely and irreversibly. However, for the purpose of this tutorial, use the ``Recycle'' option to move the deleted reference to the Recycled folder. Navigate to it and press the Space key to select it: 

\screenshot{harrit-recycled}

\noindent Let's assume that you plan to look at this paper some other time, and you want to collect references for later study in a separate folder. Navigate to the ``general interest'' folder and activate its menu (\key{Enter} or mouse double-click), then activate the option ``Add sub-folder'':%
%
\footnote{The folder context menu is quite long. In these screen shots, which show a small window, some items at the bottom are hidden. If such should be the case on your screen also, use the \arrowdown key to scroll them into view.}

\screenshot{folder-menu}

\noindent This will bring up a dialog that prompts you for the name for the new sub-folder. After entering ``read later'' and confirming, the display should look like this:

\screenshot{read-later}

\noindent The new folder has no $+$/$-$ switch before its name because it is still empty. To move the previously recycled reference into this folder, bring up the new folder's context menu and select the ``Move selected items here'' option:

\screenshot{move-here}

\noindent After completing this, you should now have the recycled reference in the new folder. (Use the \key{F2} key to display the folder's contents.)

Note that you can select more than one reference at a time; all of them will be copied or moved to a given destination. You can also select and then move or copy entire folders. As a practical exercise, create a folder named ``test'' directly below ``References''. Navigate back to ``general interest'' and select this folder with the space bar. Go back to ``test'', open its menu and select ``Copy selected items here''. 

You will now have cloned the entire folder with all of its sub-folders and references. If you would rather have all the copied references directly underneath ``test'', without any sub-folders in between, open the menu for the ``test'' folder again and select ``Flatten folder.'' This will give you the following result:

\screenshot{flatten-folder}

\noindent You will notice that, while the ``general interest'' folder is selected, the references in both ``test'' and ``general interest'' are shown in blue. If a reference appears in this shade, it indicates that \emph{this or another instance} is contained in a selected folder \emph{somewhere} in the tree---in this case, the ``general interest'' folder---and thus part of the currently active selection.%
%
\footnote{To be explicit about it: multiple instances of the same reference in different folders point to the same single instance of the reference in the database; this means that changes made to any instance will be shared by all others. That is, after all, what relational databases are good for.}
%
Deselecting ``general interest'' by hitting \key{Space} on it again will also deselect the references. 

\subsection{More on selections}




\subsection{Importing references}

References can be imported using PubMed identifiers, DOI identifiers, and \bibtex text. We will start with PubMed. Go to the PubMed website (\url{https://www.ncbi.nlm.nih.gov/pubmed/}) and type the following into the search bar:

\begin{verbatim}
blue-native[ti] anal-biochem[so]
\end{verbatim}

\noindent This will give you (at the time of this writing) 16 results. Choose the ``PMID List'' option from the page's display format control to see the identifiers for these papers. Use your mouse to select some or all of them. 

In \mbib, create a new sub-folder ``blue-native'' within ``biochemistry''. Open its menu and choose ``Import references from PubMed''. Hold down the Shift key and middle-click your mouse%
%
\footnote{This behaviour is provided by the X environment, not implemented by \mbib; it seems possible to me that some desktop environments might modify it, but I don't know for sure.}
%
to paste the selected PMIDs from the clipboard into the dialog. Click OK to start the import. The references will be imported one by one, and a unique \bibtex key will be automatically generated from the first author's last name and the year of publication. 

Notice that the imported references will also show up in the ``Recently added'' folder at the top of the tree. This can be convenient for quicker access, since references newly imported into a folder don't float to the top; they just get sorted according to either year or \bibtex key (press \key{F7} to toggle) and thus may ``get lost in the crowd'' within a folder that already contains many references.

The procedure for importing references via DOI is similar to that for PubMed identifiers; if you have a list of such identifiers, separated by white space, you can just paste them. Alternatively, you can load both types of identifiers from a plain text file by specifying the file name; \mbib will first treat the input as a file name, and failing that will attempt to use the input directly. 
 
As an example for importing \bibtex, go to a paper on 
sciencedirect (for example \href{https://doi.org/10.1016/j.bbamem.2014.05.014}{this one}) and export the \bibtex record for it. Copy it into your clipboard by mouse-selecting it, or save it to file. Open ``Import references from BibTex'' from the folder menu and then paste the \bibtex text or give the file name. 

Pasting works fine for one or a handful of references;  be warned, however, that pasting longer text will be quite slow. In this case, it is better to first save the \bibtex to file, and then enter the file name into the \bibtex import dialogue. 

\subsection{Exporting references}
\label{sec:export}

% To export individual folders to BibTex, select the appropriate item from their context menu. The same operation performed on the References folder will export the entire database. The exported file will be free of duplicates; you are free to keep copies of the same reference in as many folders as you wish. 

References can be exported to \bibtex or HTML. The formatting of the former can be fine-tuned through various settings in the \texttt{.mbib.ini} file; the latter, for the time being, cannot. Duplicates are weeded out; any folder trees are not preserved, that is, a flattened list of records is generated. 

The export operations are available in the context menu of the ``References'' folder, in which case the entire database is exported, as well as from the menu of each sub-folder, which will export the references within and below it. Additionally, export operations can also be applied to the currently selected references (including those residing in folders). Press \key{F5} to bring up a context menu that contains this and other operations pertaining to the current selection.

\subsection{Making the \mbib database available to \LaTeX{}}

Aside from exporting part or all of the \mbib database to \bibtex from an interactive session, you can also access the references you need from the command line. There are various ways to do this. 

\subsubsection{Export the references listed in the \texttt{.aux} file}

You can generate an up-to-date \bibtex file that contains all cited references (and only these%
%
\footnote{The only trick that won't work is \texttt{\textbackslash nocite\{*\}}, which is supposed to include all entries listed in the bibliography file. The best way to accommodate \texttt{\textbackslash nocite\{*\}} would be to keep all references you do want to list within a single folder inside \mbib and then to sync this folder on demand---see next section.}%
%
) on the fly by letting \mbib parse the \texttt{.aux} file of your document. Assume that in \texttt{mydocument.tex} you have declared ~\bibliography{myproject}~, which instructs \bibtex to use a file named \texttt{myproject.bib}. Then, after latexing your document, but \emph{before} running \bibtex, you can run the following command to create or update
\texttt{myproject.bib}:

\begin{verbatim}
mpalmer@rehakles:~$ mbib.sh -b auxexport -t mydocument.aux
\end{verbatim}

\noindent After that, you run \bibtex, again on the \texttt{.aux} file:

\begin{verbatim}
mpalmer@rehakles:~$ bibtex mydocument.aux
\end{verbatim}

\noindent You can also ask \mbib to first export and run \bibtex for you:

\begin{verbatim}
mpalmer@rehakles:~$ mbib.sh -b auxbibtex -t mydocument.aux
\end{verbatim}

\noindent If you include this line in your build process, \bibtex will always transparently access the current state of your database.%
%
\footnote{\mbib makes no attempt to determine if your \texttt{.bib} file is up to date (as this would require looking at all files that your main document might \texttt{\textbackslash include} or \texttt{\textbackslash input}) and just overwrites the \texttt{.bib} file every time.}
% 
Thanks to \sqlite, this is easily fast enough even with large databases and numbers of citations, and it is how I use \mbib myself with \LaTeX. 

\subsubsection{Syncing the \mbib database to a \bibtex file}

You can also export the entire database, or selected folders contained in it, to \bibtex from the command line. In the simplest case, you can use 

\begin{verbatim}
mpalmer@rehakles:~$ mbib.sh -b sync
\end{verbatim}

\noindent This will export all references in the \mbib database to a single \bibtex file, the path of which is configured in the \ini file. You can override these settings by passing additional options, for example 

\begin{verbatim}
mpalmer@rehakles:~$ mbib.sh -b sync -t myproject.bib -f myproject
\end{verbatim}

\noindent Now, only the references which, inside \mbib, are stored in the folder named ``myproject'' will be exported, and they will be saved in the file \texttt{myproject.bib}. This allows you to keep all your references in a single, central \mbib database, and at the same time use smaller, project-specific \bibtex files for your multiple separate manuscripts. A few more points to note: 

\begin{itemize}
\item If you have multiple folders named ``myproject", all of them will be exported to the same file. 

\item If the name of the folder contains spaces, you need to enclose in quotes on the command line:

\begin{verbatim}
mpalmer@rehakles:~$ mbib.sh -b sync -t myproject.bib -f "my project"
\end{verbatim}

\item The \texttt{\mbib\ -b sync} command will only export the database if its time stamp is more recent than that on the \bibtex file (or if the latter does not yet exist, obviously). 

\end{itemize}

\noindent The last point means that, regardless of file size, it costs very little to run this command routinely during document compilation. Therefore, this is another method to ensure that your document gets compiled against the current state of the \mbib database.%
%
\footnote{Of course, if you do the full database dump, \bibtex may have to scan through a very large \texttt{.bib} file. While \bibtex itself is very fast even with large files, other tools like \texttt{biber} reportedly can be noticeably slower.}

\subsection{Using \mbib with \loo}

External programs can communicate with \loo via either sockets or TCP. \mbib supports both variants. Here, I will illustrate the set-up and program I use on my own laptop. I use LibreOffice, but this should work the same with OpenOffice, too. 

To make LibreOffice listen to connection attempts, I have configured my (KDE) desktop menu entry for LibreOffice to use the command

\begin{verbatim}
libreoffice  --accept="pipe,name=abraxas;urp" %U
\end{verbatim}

The \texttt{\%U} argument variable will be substituted by KDE with a file name to open as applicable. If you start LibreOffice from a command line, use the actual file name instead, or omit it and then open a text file through the menu. 

If you left the default settings inside the \ini file in force, then \mbib should now be able to insert citations into an open \loo document. Select some references (by hitting \key{Space} on them). Bring up the selection context menu with \key{F5} and select ``Cite in OpenOffice.'' The selected references should now show up in the text document:

\screenshot[clip,trim=0 0 3px 0]{cite-ooo}

\noindent The above procedure inserts the citations into the document in \loo{}'s own format, as if they had been selected from its own internal database. Accordingly, you can now create a bibliography using \loo{}'s menu `Insert \arrowright TOC and Index \arrowright TOC, Index or Bibliography,' and then update%
%
\footnote{To update the bibliography after inserting more citations, right-click on the bibliography heading and select `update index' from the menu.}
%
and format it using \loo's built-in functions. 

While \loo{}'s own bibliography may be good enough for your needs, tweaking its appearance can be a bit limiting. As an alternative, you can use \jabref to generate and style your bibliography. To this end, you first have to change the inserted citations to a format that \jabref can understand. Assume that you have saved the document as \texttt{mydoc.odt}. Then, on the command line, invoke 

\begin{verbatim}
mpalmer@rehakles:~$ mbib.sh -b jabrefy -t mydoc.odt
\end{verbatim}

\noindent This will produce two files: \texttt{mydoc\_jr.odt} and \texttt{mydoc\_jr.bib}. The first will contain the same citation entries as the original \texttt{mydoc.odt}, but now formatted for \jabref. The second file contains the \bibtex items that match the citations in the \texttt{.odt} file, exported from the \mbib database.  

Actually formatting the bibliography requires that you open \texttt{mydocs\_jr.odt} in \loo, and \texttt{mydocs\_jr.bib} in \jabref. Connect from \jabref to \loo (press the button with red circle in screen shot below), and refresh the document (button with blue circle). 

\screenshot{jabref}

\noindent This should generate the bibliography; you can then use \jabref's facilities for defining and applying styles. If you had previously created a bibliography in \loo directly, you will now have two bibliographies; get rid of the \loo one by right-clicking on its heading and select `delete index' from the context menu.

While this process of formatting the bibliography by first running the \mbib batch command and then \jabref is a bit clunky. However, it really needs to be applied only once, when you have completed the document. Also note that \jabref{}-formatting is ``round-trippable''---you can insert additional references into an already \jabref-formatted document. When you again \texttt{-b jabrefy} it and then reformat it in \jabref, the newly inserted references will be properly merged with the old ones. 

\subsection{Inserting citations into \LaTeX\ documents from within \mbib}

So far, I have only figured out how to push keys from \mbib to Texmaker and TexStudio; the context menus for individual references and for selections (\key{F5}) contain a `Cite in LaTeX' item for this purpose. For other editors, the only option is to copy the keys to the X clipboard and then paste them. I'd be happy to add support for other editors if anyone can point me to the technical details on IPC for these.

\subsection{Searching and filtering}
\label{sec-searching}

One important task we have not yet touched upon is searching for references. Navigate to the ``Search'' folder, bring up its context menu, and select ``New search''. This will bring up an empty search mask. Into the \texttt{abstract} field, type ``synthesis'', and hit \texttt{OK}. You should now see the following screen:

\screenshot[clip,trim=0 1.2in 0 0]{search-results}

\noindent The search results have been appended to the ``Search'' folder. You can work with them the usual way---edit them, select, copy, and move them. You cannot delete them individually, because it would be inconsequential, as they are merely copies of instances stored elsewhere. However, you can use ``Delete from all folders'', in which case those other instances will be deleted also, and the references will then show up in the Trash. 

If you press \key{F2} to bring up the details view for either reference in the Search Folder, you will see an ``also in''
%
\label{pg-also-in}%
%
entry that is a hyperlink. Following it will take you to the directory in which this record resides. 

\subsubsection{Multiple search terms}

You can modify your search by selecting ``Edit last search'' from the ``Search'' folder's context menu. If you fill in multiple fields at the same time, all constraints will be applied in conjunction (``and''). You can also apply multiple constraints to a single field using boolean operators. For example, typing the phrase: ~respiratory || cytochrome && PAGE~ into the ``abstract'' field of the search mask will look for records that contain the words ``respiratory'' or ``cytochrome'', as well as ``PAGE'' in the abstract. Note that

\begin{itemize}
\item ~||~ and ~&&~ represent the boolean \texttt{or} and \texttt{and} operators, respectively.
\item ~||~ takes precedence over (joins more strongly than) ~&&~. While this deviates from convention in mathematics, I find it more useful in database queries; it pretty much does away with the need for using brackets at all (which \mbib anyway doesn't support at this time). You can, however, change the precedence in the \ini file if you wish.
\end{itemize}

\subsubsection{Finding folders}

There is no proper search facility for folders. However, there are two aids for finding your way through the brushwork:

\begin{itemize}
\item You can filter folders by name. Press \key{F9} to enter a word or phrase to search for. Any folder whose name contains the entered phrase, as well as its ancestors and descendants, will be displayed, while all other regular folders (those below ``References'') will be hidden. Press \key{F10} to revert to the full display. 

\item You can also hide all references from view, displaying only the folders, using \key{F8}. (If you do so, folders that contain only references but no sub-folders will be displayed without the \texttt{+/-} expansion marker before their names.) This condensed view lets you explore the folder tree more speedily. Press \key{F8} again to revert to the full display. 
\end{itemize}

\noindent Finally, if you remember neither the name nor the location of the folder you are looking for, it may be useful to search for a pertinent reference instead, display its details (\key{F2}), and then follow the ``Also in'' link to its location in the tree (see section \ref{sec-searching}, page \pageref{pg-also-in}). 

\subsection{Viewing PDF files}

If you select ``Show PDF file'' from a reference's context menu, \mbib will look for a PDF file whose name matches its \bibtex key; e.g.\ the PDF for the record \texttt{Harrit2009} should be named \texttt{Harrit2009.pdf}. Furthermore, by default, \mbib looks for PDF files only below the \texttt{data/pdf/} directory that itself resides in the \mbib installation directory.%
%
\footnote{If you don't want to move your PDF files there, you can change the path in the \ini file, or you can create symlinks from inside \texttt{data/pdf} to your actual storage locations.}

If \mbib fails to find the requested file in the preconfigured directory, it can use the \texttt{locate} program%
%
\footnote{On Debian, \texttt{locate} resides in the package of the same name. It requires the \texttt{updatedb} command to be run periodically in order to catch up with added or deleted files on your system.}
%
to find it elsewhere. This feature is disabled by default, but you can enable it in the \ini file (change option \texttt{pdf\_locate = false} to \texttt{true}). This may be convenient if you prefer to store your PDF files not in a single centralized location but closer to the various projects and manuscripts you may be working on. 

\subsection{Multiple databases}

One of my key objectives with \mbib was to organize all my references within in a single database. However, if you prefer, you can have multiple databases and select the one to use from the command line: 

\begin{verbatim}
mpalmer@rehakles:~$ mbib.sh -d my/wonderful/database.sqlite
\end{verbatim}

\noindent This also works in conjunction with the various other command line operations that were described above in this tutorial. 

\begin{comment}
\section{Running \mbib}

The program runs inside a console window. Assuming you have put a start-up script into your shell's path, just open a shell window (e.g.\ \texttt{konsole} on KDE, or \texttt{xterm} on any X-based desktop) and run \texttt{mbib.sh}.

%First, the standard way, and then maybe the wrapper bash script. Also how to add it to the K Menu and the work-around for konsole - use xterm instead. Customization of xterm via the ~/.Xresources file and 
%
%xrdb -merge ~/.Xresources
%
%Leave it to others to contribute instructions on how to run it on other desktops. 




\subsection{The user interface}

\begin{itemize}
\item navigation (mouse, keyboard)
\item viewing and editing references
\item importing references
\item moving things around (using selections)
\item searching and filtering
\item using the clipboard (requires xclip)
\item 
\end{itemize}

\end{comment}

\section{Reference} 

This section will---some day, far, far into the future, maybe about the same time that the U.S.\ and Israeli governments finally come clean about 9/11---describe everything, including configuration and code structure. 

\section{Some gotchas}

Here are some things to keep in mind. 

\begin{itemize}

%\item Item selections are stored in the database. Maybe we should offer a configuration option to deselect everything, just as we do with the ``recently added'' folder. Yes, we do that now. 

\item All changes to the data---imports, edits, deletions---are saved to the database immediately, and there is no undo function. (However, as discussed in section \ref{sec:moving}, deleted references can still be retrieved from the ``Trash'' folder.)

\item On a related note: \mbib, by default, writes a backup version of the \sqlite file (with the appended extension \texttt{.bak}) next to the current database. (You can disable automatic backups in the \ini file.)  

\emph{The backup file gets overwritten at program start if it deviates from the current database file.} 

Therefore, if you want to revert to the backup file, exit \mbib and then replace the current database file with the backup file \emph{before} restarting \mbib.

% \item Moving selected items from search results: the search folder is basically a simple folder in the database. So, if items in this folder are selected and moved, the copies in the permanent folder stay where they are. this is discussed in the tutorial.

%We may think about an option to erase all other copies of a reference. Yes, we have that now. Or all selected references -- we don't have that yet. If we did have it, I guess we should offer a confirmation dialog that specifies the number of selected references.

\item All the special top-level folders are special and protected just inside \mbib, because specific operations are available and others excluded; in particular, they cannot be deleted, moved, or renamed. However, no such protective restrictions apply when working with the database using an \sqlite shell. 

Generally speaking, it is a good idea to make a backup copy of the database file and stash it away in a safe place before performing major surgery in SQL. 
\end{itemize}



\end{document}